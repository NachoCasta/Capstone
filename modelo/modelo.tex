\documentclass[12pt]{article}

\usepackage{fullpage}
\usepackage{graphicx}
\usepackage{amssymb}
\usepackage{amsmath}
\usepackage[none]{hyphenat}
\usepackage{parskip}
\usepackage[spanish]{babel}
\usepackage[utf8]{inputenc}
\usepackage{hyperref}
\usepackage{fancyhdr}
\usepackage{tabto}
\usepackage{multicol}
\setlength{\headheight}{15.2pt}
\setlength{\headsep}{5pt}
\setlength{\columnseprule}{1pt}
\pagestyle{fancy}

\newcommand{\R}{\mathbb{R}}
\newcommand{\N}{\mathbb{N}}
\newcommand{\Exp}[1]{\mathcal{E}_{#1}}
\newcommand{\List}[1]{\mathcal{L}_{#1}}
\newcommand{\EN}{\Exp{\N}}
\newcommand{\LN}{\List{\N}}

\makeatletter
\newcommand*\bigcdot{\mathpalette\bigcdot@{1}}
\newcommand*\bigcdot@[2]{\mathbin{\vcenter{\hbox{\scalebox{#2}{$\m@th#1\bullet$}}}}}
\makeatother

\newcommand{\comment}[1]{}
\newcommand{\lb}{\\~\\}
\newcommand{\eop}{_{\square}}
\newcommand{\hsig}{\hat{\sigma}}
\newcommand{\ra}{\rightarrow}
\newcommand{\lra}{\leftrightarrow}

\newcommand{\grupo}{Grupo 03}

% Problema

\newcommand{\problema}{Modelo de distribución dentro de una feria} 

%

\rhead{\grupo \ - \ \problema}

\begin{document}
% Titulo
\begin{center}
\vspace{0.5cm}
{\large\bf \problema}\\
\vspace{0.2cm}
\footnotesize{\grupo}
\end{center}

\subsection*{En base a una feria fija}

\subsubsection*{Variables}
\begin{itemize}
	\item[] $x_{l,t} = \{1$ ssi el local $l$ es del tipo $t\}$
	\item[] $y_{c,p,j,f} = \{1$ ssi la calle $c$ compra en el puesto $p$ del del lado $j$ de la calle $f\}$
	\item[] $z_{p,j,f,l} = \{1$ ssi el puesto $p$ del lado $j$ de la calle $f$ pertenece al local $l\}$
\end{itemize}

\subsubsection*{Conjuntos}
\begin{itemize}
	\item[] $F \ra$ Calles de feria $\{1, \dots , g\}$
	\item[] $L \ra$ Locales $\{1, \dots , h\}$
	\item[] $P_{f,j} \ra$ Puestos de la calle $f$ de la feria en el lado $j$ de la calle $\{1, \dots , 13\}$
	\item[] $T \ra$ Tipos de locales $\{1, \dots , k\}$
	\item[] $C \ra$ Calles $\{1, \dots , m\}$
	\item[] $D \ra$ Dias $\{1, \dots , 7\}$
\end{itemize}

\subsubsection*{Parametros}
\begin{itemize}
	\item[] $u_{p,j,f} \ra$ posición $(x, y)$ del puesto $p$ del lado $j$ de la calle $f$
	\item[] $v_c \ra$ posición $(x, y)$ de la calle $c$
	\item[] $a_c \ra$  distancia de la calle $c$ al supermercado
	\item[] $d_{c,t,d} \ra$ demanda de la calle $c$ por el tipo de local $t$ en el dia $d$
	\item[] $e_t \ra$  cantidad de puestos que ocupa un local de tipo $t$
	\item[] $t_t \ra$  cantidad de puestos totales del tipo de local $t$
	\item[] $s \ra$ sensibilidad logit
\end{itemize}

\subsubsection*{Función Objetivo}
\begin{itemize}
\item[] $d = \max\limits_{c \in C, f \in F, j \in \{1,2\}, p \in P_f}\{ y_{c,p,j,f} |u_{p,j,f} - v_c|_1^1 \} \ra$ distancia de la calle $c$ al puesto más lejano que irá\\
\item[]$\max 
\displaystyle\sum\limits_{d \in D}
\sum\limits_{c \in C}
\sum\limits_{f \in F} 
\sum\limits_{j \in \{1,2\}} 
\sum\limits_{p \in P_f} 
y_{c,p,j,f} x_{l,t} z_{p,j,l,f} \left( \dfrac{e^{sd}}{e^{sd} + e^{sa_c}}\right)$
\end{itemize}
\subsubsection*{Restricciones}

\begin{itemize}
    \item Cada calle debe ir a un puesto de cada tipo
    
    $\sum\limits_{f \in F} \sum\limits_{j \in \{1,2\}} \sum\limits_{p \in P_f} \sum\limits_{t \in T} y_{c,p,j,f} x_{l,t} z_{p,j,l,f} = 1$ \tab $\forall c \in C$ 
    
    \item La cantidad de puestos equivalentes totales utilizados por cada tipo de local debe ser igual a la indicada
    
    $\sum\limits_{f \in F} \sum\limits_{j \in \{1,2\}} \sum\limits_{p \in P_f} x_{l,t} z_{p,j,l,f} = t_t$ \tab $\forall t \in T$ 
    
    \item La cantidad de puestos equivalentes totales utilizados por cada tipo de local debe ser igual a la indicada
    
    $\sum\limits_{f \in F} \sum\limits_{j \in \{1,2\}} \sum\limits_{p \in P_f} x_{l,t} z_{p,j,l,f} = t_t$ \tab $\forall t \in T$
    
    \item Cada local tiene la cantidad de puestos que le corresponde, según el tipo de local
    
    $\sum\limits_{p \in P_f} x_{l,t} z_{p,j,f,l} = x_{l,t} e_t $ \tab $\forall l \in L, \quad \forall t \in T, \quad \forall f \in F, \quad \forall j \in \{1,2\}$ \\
    
    $\sum\limits_{f \in F} \sum\limits_{j \in \{1,2\}} \sum\limits_{p \in P_f} x_{l,t} z_{p,j,f,l} = x_{l,t}e_t $ \tab $\forall l \in L, \quad \forall t \in T$  
    
    \item Cada local debe tener sus puestos adyacentes
    
    $|p_1 - p_2|z_{p_1,j,f,l} z_{p_2,j,f,l} x_{l,t} < e_t $ \tab $\forall p_1, p_2 \in P_f, \quad \forall f \in F, \quad \forall t \in T, \quad \forall l \in L$ 
    
    \item Dos puestos del mismo tipo no pueden estar al frente
    
    $\sum\limits_{l \in L} x_{l,t} z_{p,1,f,l} \neq \sum\limits_{l \in L} x_{l,t} z_{p,2,f,l} $ \tab $\forall p \in P_f, \quad \forall f \in F, \quad \forall t \in T$ 
    
    \item Dos puestos del mismo tipo no pueden en diagonal
    
    $\sum\limits_{l \in L} x_{l,t} z_{p,1,f,l} \neq \sum\limits_{l \in L} x_{l,t} z_{p-1,2,f,l} $ \tab $\forall p > 1 \in P_f, \quad \forall f \in F, \quad \forall t \in T$\\
    
    $\sum\limits_{l \in L} x_{l,t} z_{p,1,f,l} \neq \sum\limits_{l \in L} x_{l,t} z_{p+1,2,f,l} $ \tab $\forall p < 13 \in P_f, \quad \forall f \in F, \quad \forall t \in T$
    
    \item Dos locales adyacentes no pueden ser iguales
    
    \small$\sum\limits_{l \in L} z_{p,j,f,l} z_{p-1,j,f,l} = \sum\limits_{t \in T}\left( \sum\limits_{l \in L} x_{l,t} z_{p,j,f,l} \sum\limits_{l \in L} x_{l,t} z_{p-1,j,f,l}\right) \quad \forall p > 1\in P_f, \ \forall f \in F, \ \forall j \in \{1,2\}$ \\
    
    \small$\sum\limits_{l \in L} z_{p,j,f,l} z_{p+1,j,f,l} = \sum\limits_{t \in T}\left( \sum\limits_{l \in L} x_{l,t} z_{p,j,f,l} \sum\limits_{l \in L} x_{l,t} z_{p+1,j,f,l}\right) \quad \forall p < 13\in P_f, \ \forall f \in F, \ \forall j \in \{1,2\}$ 
    
    \item Naturaleza de las variables
    
    $x_{l,t} \in \{0, 1\}$

    $y_{c, p, j,f} \in \{0, 1\}$
    
    $z_{p, j,f,l} \in \{0, 1\}$
    
\end{itemize}
    
% Fin del documento
\end{document}
